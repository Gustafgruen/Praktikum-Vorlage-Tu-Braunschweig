\documentclass[]{Praktikum}

\def\VTag{01}
\def\VMonat{01}
\def\VJahr{1970}
\VNummer = 21

\begin{document}
\section{Einleitung}
\lipsum[2]

\section{Physikalische Grundlagen}
\subsection{Grundlage 1}
\lipsum[1-3]

\subsection{Grundlage 2}
\lipsum[1-3]

\subsection{Grundlage 3}
\lipsum[1-3]

\subsection{Grundlage 4}
\lipsum[1-3]

\newpage
\section{Versuchsaufbau}
\lipsum[4]

\section{Versuchsteil 1}
\subsection{Versuchsdurchführung}
\lipsum[4]

\subsection{Auswertung und Diskussion}
\lipsum[2-3]

\section{Versuchsteil 2}
\subsection{Versuchsdurchführung}
\lipsum[4]

\subsection{Auswertung und Diskussion}
\lipsum[2-3]

\section{Versuchsteil 3}
\subsection{Versuchsdurchführung}
\lipsum[4]

\subsection{Auswertung und Diskussion}
\lipsum[2-3]

\section{Fazit}
\lipsum[1]

\newpage
\appendix
\addappheadtotoc
\section{Fehler}
Beim Messen, mit dem IKEA-Maßband gibt es einen Fehler von $\Delta l=0,5\mmu$.

\section{Messdaten}
\subsection{Versuchsteil 1}
\begin{table}[H]
    \centering
    \begin{tabular}{|c|c|}
        \hline
        Index & Daten \\\hline
        0 & 50 \\
        1 & 3 \\
        2 & 45\\\hline
    \end{tabular}
    \caption{Messdaten 1}
    \label{tab: MD1}
\end{table}

\subsection{Versuchsteil 2}
\begin{table}[H]
    \centering
    \begin{tabular}{|c|c|}
        \hline
        Index & Daten \\\hline
        0 & 50 \\
        1 & 3 \\
        2 & 45\\\hline
    \end{tabular}
    \caption{Messdaten 2}
    \label{tab: MD2}
\end{table}

\subsection{Versuchsteil 3}
\begin{table}[H]
    \centering
    \begin{tabular}{|c|c|}
        \hline
        Index & Daten \\\hline
        0 & 50 \\
        1 & 3 \\
        2 & 45\\\hline
    \end{tabular}
    \caption{Messdaten 3}
    \label{tab: MD3}
\end{table}
\printbibliography
\end{document}
